% Options for packages loaded elsewhere
\PassOptionsToPackage{unicode}{hyperref}
\PassOptionsToPackage{hyphens}{url}
%
\documentclass[
]{article}
\usepackage{lmodern}
\usepackage{amssymb,amsmath}
\usepackage{ifxetex,ifluatex}
\ifnum 0\ifxetex 1\fi\ifluatex 1\fi=0 % if pdftex
  \usepackage[T1]{fontenc}
  \usepackage[utf8]{inputenc}
  \usepackage{textcomp} % provide euro and other symbols
\else % if luatex or xetex
  \usepackage{unicode-math}
  \defaultfontfeatures{Scale=MatchLowercase}
  \defaultfontfeatures[\rmfamily]{Ligatures=TeX,Scale=1}
\fi
% Use upquote if available, for straight quotes in verbatim environments
\IfFileExists{upquote.sty}{\usepackage{upquote}}{}
\IfFileExists{microtype.sty}{% use microtype if available
  \usepackage[]{microtype}
  \UseMicrotypeSet[protrusion]{basicmath} % disable protrusion for tt fonts
}{}
\makeatletter
\@ifundefined{KOMAClassName}{% if non-KOMA class
  \IfFileExists{parskip.sty}{%
    \usepackage{parskip}
  }{% else
    \setlength{\parindent}{0pt}
    \setlength{\parskip}{6pt plus 2pt minus 1pt}}
}{% if KOMA class
  \KOMAoptions{parskip=half}}
\makeatother
\usepackage{xcolor}
\IfFileExists{xurl.sty}{\usepackage{xurl}}{} % add URL line breaks if available
\IfFileExists{bookmark.sty}{\usepackage{bookmark}}{\usepackage{hyperref}}
\hypersetup{
  pdftitle={Data III},
  pdfauthor={Sunny Chou},
  hidelinks,
  pdfcreator={LaTeX via pandoc}}
\urlstyle{same} % disable monospaced font for URLs
\usepackage[margin=1in]{geometry}
\usepackage{graphicx,grffile}
\makeatletter
\def\maxwidth{\ifdim\Gin@nat@width>\linewidth\linewidth\else\Gin@nat@width\fi}
\def\maxheight{\ifdim\Gin@nat@height>\textheight\textheight\else\Gin@nat@height\fi}
\makeatother
% Scale images if necessary, so that they will not overflow the page
% margins by default, and it is still possible to overwrite the defaults
% using explicit options in \includegraphics[width, height, ...]{}
\setkeys{Gin}{width=\maxwidth,height=\maxheight,keepaspectratio}
% Set default figure placement to htbp
\makeatletter
\def\fps@figure{htbp}
\makeatother
\setlength{\emergencystretch}{3em} % prevent overfull lines
\providecommand{\tightlist}{%
  \setlength{\itemsep}{0pt}\setlength{\parskip}{0pt}}
\setcounter{secnumdepth}{-\maxdimen} % remove section numbering

\title{Data III}
\author{Sunny Chou}
\date{1/25/2020}

\begin{document}
\maketitle

\hypertarget{reading-data-40-points}{%
\subsection{Reading data (40 points)}\label{reading-data-40-points}}

First, we need to read the data into R. For this assignment, I ask you
to use data from the youth self-completion questionnaire (completed by
children between 10 and 15 years old) from Wave 9 of the Understanding
Society. It is one of the files you have downloaded as part of SN6614
from the UK Data Service. To help you find and understand this file you
will need the following documents:

\begin{enumerate}
\def\labelenumi{\arabic{enumi})}
\tightlist
\item
  The Understanding Society Waves 1-9 User Guide:
  \url{https://www.understandingsociety.ac.uk/sites/default/files/downloads/documentation/mainstage/user-guides/mainstage-user-guide.pdf}
\item
  The youth self-completion questionnaire from Wave 9:
  \url{https://www.understandingsociety.ac.uk/sites/default/files/downloads/documentation/mainstage/questionnaire/wave-9/w9-gb-youth-self-completion-questionnaire.pdf}
\item
  The codebook for the file:
  \url{https://www.understandingsociety.ac.uk/documentation/mainstage/dataset-documentation/datafile/youth/wave/9}
\end{enumerate}

\begin{verbatim}
## -- Attaching packages -------------------------------------------------------------------------------------------------------------------------------- tidyverse 1.3.0 --
\end{verbatim}

\begin{verbatim}
## v ggplot2 3.2.1     v purrr   0.3.3
## v tibble  2.1.3     v dplyr   0.8.3
## v tidyr   1.0.0     v stringr 1.4.0
## v readr   1.3.1     v forcats 0.4.0
\end{verbatim}

\begin{verbatim}
## -- Conflicts ----------------------------------------------------------------------------------------------------------------------------------- tidyverse_conflicts() --
## x dplyr::filter() masks stats::filter()
## x dplyr::lag()    masks stats::lag()
\end{verbatim}

\begin{verbatim}
## Parsed with column specification:
## cols(
##   .default = col_double()
## )
\end{verbatim}

\begin{verbatim}
## See spec(...) for full column specifications.
\end{verbatim}

\hypertarget{tabulate-variables-10-points}{%
\subsection{Tabulate variables (10
points)}\label{tabulate-variables-10-points}}

In the survey children were asked the following question: ``Do you have
a social media profile or account on any sites or apps?''. In this
assignment we want to explore how the probability of having an account
on social media depends on children's age and gender.

Tabulate three variables: children's gender, age (please use derived
variables) and having an account on social media.

\begin{verbatim}
## , , ysma = 1
## 
##    yage
## yg    9  10  11  12  13  14  15  16  17
##   1   1  19  87 152 199 204 216 161  58
##   2   0  27 114 164 193 237 236 160  49
## 
## , , ysma = 2
## 
##    yage
## yg    9  10  11  12  13  14  15  16  17
##   1   0  29  88  86  50  19  19  11   4
##   2   0  33  66  59  37  17  10   2   0
\end{verbatim}

\hypertarget{recode-variables-10-points}{%
\subsection{Recode variables (10
points)}\label{recode-variables-10-points}}

We want to create a new binary variable for having an account on social
media so that 1 means ``yes'', 0 means ``no'', and all missing values
are coded as NA. We also want to recode gender into a new variable with
the values ``male'' and ``female'' (this can be a character vector or a
factor).

\begin{verbatim}
##    Min. 1st Qu.  Median    Mean 3rd Qu.    Max.    NA's 
##   1.000   1.000   1.000   1.189   1.000   2.000      14
\end{verbatim}

\begin{verbatim}
## newysma
##    0    1 
##  530 2277
\end{verbatim}

\begin{verbatim}
## newysma
##    0    1 
##  530 2277
\end{verbatim}

\begin{verbatim}
## gender
## female   male 
##   1410   1411
\end{verbatim}

\hypertarget{calculate-means-10-points}{%
\subsection{Calculate means (10
points)}\label{calculate-means-10-points}}

Produce code that calculates probabilities of having an account on
social media (i.e.~the mean of your new binary variable produced in the
previous problem) by age and gender.

\begin{verbatim}
## 
##    0    1 
##  530 2277
\end{verbatim}

\begin{verbatim}
##        
##            female      male
##   FALSE 0.0798005 0.1090132
##   TRUE  0.4203776 0.3908087
\end{verbatim}

\begin{verbatim}
##        
##                   10           11           12           13           14
##   FALSE 0.0220876380 0.0548628429 0.0516565729 0.0309939437 0.0128250802
##   TRUE  0.0163876024 0.0716066975 0.1125757036 0.1396508728 0.1571072319
##        
##                   15           16           17            9
##   FALSE 0.0103313146 0.0046312789 0.0014250089 0.0000000000
##   TRUE  0.1610260064 0.1143569647 0.0381189882 0.0003562522
\end{verbatim}

\hypertarget{write-short-interpretation-10-points}{%
\subsection{Write short interpretation (10
points)}\label{write-short-interpretation-10-points}}

Females have a 42\% of probability of having a social media account
while males have 39.1\%, the differences do not seem to be significant.
Whereas the presence of social media accounts has a rising trend from 9
to 15 and declines afterwards.

\hypertarget{visualise-results-20-points}{%
\subsection{Visualise results (20
points)}\label{visualise-results-20-points}}

Create a statistical graph (only one, but it can be faceted)
illustrating your results (i.e.~showing how the probability of having an
account on social media changes with age and gender). Which type of
statistical graph would be most appropriate for this?

\begin{verbatim}
## , ,  = female
## 
##        
##                   10           11           12           13           14
##   FALSE 0.0117563235 0.0235126470 0.0210188814 0.0131813324 0.0060562879
##   TRUE  0.0096188101 0.0406127538 0.0584253652 0.0687566797 0.0844317777
##        
##                   15           16           17            9
##   FALSE 0.0035625223 0.0007125045 0.0000000000 0.0000000000
##   TRUE  0.0840755255 0.0570003563 0.0174563591 0.0000000000
## 
## , ,  = male
## 
##        
##                   10           11           12           13           14
##   FALSE 0.0103313146 0.0313501959 0.0306376915 0.0178126113 0.0067687923
##   TRUE  0.0067687923 0.0309939437 0.0541503384 0.0708941931 0.0726754542
##        
##                   15           16           17            9
##   FALSE 0.0067687923 0.0039187745 0.0014250089 0.0000000000
##   TRUE  0.0769504809 0.0573566085 0.0206626291 0.0003562522
\end{verbatim}

\includegraphics{testAssignment_files/figure-latex/unnamed-chunk-7-1.pdf}

\begin{verbatim}
## integer(0)
\end{verbatim}

\end{document}
